\chapter{Diverse}

\section{Vedtekter}
Vedtektene er Casual Gamings øverste regelverk og kan kun endres ved $\nicefrac{2}{3}$-flertall ved et årsmøte. Forslag til endring av vedtektene må være sendt til Styret senest en uke før et årsmøte. En vedtektsendring trer normal ikke i kraft før etter årsmøtet den vedtas på er hevet, men kan tres i kraft umiddelbart under Årsmøtet ved avstemning med $\nicefrac{3}{4}$-flertall. Styret kan gjøre redaksjonelle endringer i vedtektene ved $\nicefrac{2}{3}$-flertall på ordinært styremøte, så lenge dette ikke endrer betydningen av paragrafen det endres på.

\section{Mislighold av verv}
Om en innehaver av et verv misligholder sine arbeidsoppgaver, kan ethvert medlem av Casual Gaming stille mistillitsforslag ovenfor vedkommende. Mistillitsforslaget skal leveres skriftlig til Styret, som skal behandle saken. Ved mistillitsforslag mot et styremedlem blir den anklagede umiddelbart suspendert inntil Styret har kommet med en avgjørelse. Mistillitsforslaget leses opp i Styret, deretter skal den anklagede få en mulighet til å forsvare seg før Styret diskuterer og avgjør saken uten den anklagede til stede. For å beholde et beslutningsdyktig og fungerende Styre vil det kun være mulig å stille mistillitsforslag mot ett styremedlem av gangen. Styret har to uker på å behandle et mistillitsforslag.

\section{Alminnelig flertall}
Alle vedtak i Casual Gamings organer krever alminnelig flertall hvis ikke annet er spesifisert i vedtektene.

\section{Oppløsning}
Casual Gaming kan bare oppløses med $\nicefrac{3}{4}$-flertall på et ordinært årsmøte og et påfølgende ekstraordinært årsmøte hvor oppløsning er eneste sak som skal behandles. Ved oppløsning bestemmer Årsmøtet fordelingen av Casual Gamings midler. Midlene som blir fordelt må gå tilbake til støtteorganene de kom fra eller til fordel for interesser som har vært sentrale hos Casual Gaming.

\section{Medlemsår og regnskapsår}
Medlemsår og regnskapsår skal følge kalenderåret.
