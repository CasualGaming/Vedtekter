\chapter{Årsmøtet}
Årsmøtet er Casual Gamings øverste organ og skal avholdes hvert år. Styret innkaller til årsmøte med minst to ukers varsel. På årsmøtet deltar alle medlemmer som har betalt medlemskontingenten i forkant av møtet, med tale-, forslags‐ og stemmerett. For at årsmøtet skal ha vedtaksdyktighet må minst 10 medlemmer møte opp. Følgende saker skal alltid tas opp på årsmøtet:
\begin{enumerate}
    \item Valg av ordstyrer, tellekorps, paraferer og referent.
    \item Årsmeldinger fra styret.
    \item Vedtektsendringer.
    \item Gjennomgang av regnskap for forrige år.
    \item Godkjenning av budsjett for inneværende år.
    \item Valg av Styrets medlemmer.
    \item Tiltaksplan for inneværende år.
\end{enumerate}

\section{Frister}
\begin{itemize}
    \item Orientering om planlagt årsmøte må sendes ut minst to uker i forveien.
    \item En uke før møtet sendes det innkalling med endelig sakliste.
    \item Forslag til vedtektsendring må sendes inn til sittende styre minst en uke før årsmøte.
\end{itemize}

\section{Valg}
Medlemmene av styret–med unntak av økonomiansvarlig–velges inn av årsmøtet. Alle styrets valgte medlemmer–med unntak av leder og nestleder–velges samtidig til leder av undergruppen vedkommende stiller til valg for. For å kunne bli valgt til styreverv må man være medlem i Casual Gaming. Ved styrevalg må en kandidat ha alminnelig flertall av tilstedeværende stemmeberettslige. Hvis dette ikke er tilfelle, fjernes motkandidaten med minst stemmer, og valget gjentas uten nevnt kandidat. Hvis det gjenstår kun en kandidat, som ikke får alminnelig flertall, skal det gjennomføres et og kun et gjenvalg bestående kun av denne kandidaten.

\section{Ekstraordinære årsmøter}
Det avholdes ekstraordinært årsmøte hvis alminnelig flertall i styret eller minst 1/4 av medlemmene krever det. Innkalling må sendes med minst en ukes varsel, og kun sakene nevnt i innkallingen kan tas opp.
