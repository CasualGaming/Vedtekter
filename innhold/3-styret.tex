\chapter{Styret}
Styret er det øverste organet i foreningen mellom årsmøtene. Styrets ansvarsområde går på å drive Casual Gaming som organisasjon mellom årsmøtene.

\section{Styrets oppbygning}
Styret består av følgende styreverv:
\begin{enumerate}[a.]
    \item Leder
    \item Nestleder
    \item Økonomiansvarlig, ved behov
    \item Opptil 5 ekstra medlemmer
    \item Opptil 2 varamedlemmer
\end{enumerate}

\section{Styreverv}
Et årsmøtevalgt styremedlem innehar sitt verv frem til neste ordinære årsmøte. Styret har fullmakt til å omfordele styreverv og å supplere seg selv ved ledige styreverv, med unntak av vervene leder og nestleder. Leder og nestleder kan kun velges av årsmøtet. Dersom vervet som leder er ledig og det finnes en nestleder, skal nestleder umiddelbart rykke opp til vervet leder med midlertidig virkning frem til neste årsmøte.

\section{Varaverv}
Dersom et styreverv står ledig, med unntak av leder, nestleder og økonomiansvarlig, og det finnes et varamedlem, skal varamedlemmet umiddelbart rykke opp til vervet med midlertidig virkning frem til neste ordinære årsmøte eller til styret har funnet en ny person til vervet.

\section{Styremøter}
Det avholdes styremøte dersom styreleder, styrenestleder eller minst to styremedlemmer kaller inn til det. Styret er vedtaksdyktig under styremøter når \nicefrac{3}{5} av styrets medlemmer og styreleder eller styrenestleder er tilstede. Ved saksbehandling gjelder absolutt flertall av alle styrets medlemmer. Ved stemmelikhet har leder dobbeltstemme.

\section{Distribuert saksbehandling}
Styret kan foruten styremøter behandle saker gjennom distribuert saksbehandling. Styret er vedtaksdyktig under distribuert saksbehandling dersom alle styremedlemmer er varslet og minst \nicefrac{3}{4} av styrets medlemmer har avgitt stemmen sin innen 24 timer. Dersom noen av styrets medlemmer krever det så skal saken utsettes til neste styremøte.

\section{Signaturrett}
Styreleder, styrenestleder og økonomiansvarlig har signaturrett for foreningen hver for seg. Styret kan også tildele andre personer prokura.
