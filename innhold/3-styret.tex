\chapter{Styret}
Styret er det øverste organet i foreningen mellom årsmøtene. Styrets ansvarsområde går på å drive Casual Gaming som organisasjon mellom årsmøtene.

\section{Styrets oppbygning}
Styret består av følgende styreverv:
\begin{enumerate}[a.]
    \item Leder
    \item Nestleder
    \item Økonomiansvarlig, ved behov
    \item Opptil 5 ekstra medlemmer
\end{enumerate}

\section{Styreverv}
Et styremedlem innehar sitt verv til ny person har blitt valgt til stillingen. Styret har fullmakt til å omfordele styreverv og å supplere seg selv ved ledige styreverv. Dersom et styreverv står ledig, skal Styret internt tildele ansvaret for vervet innad i Styret. Dersom vervet som styreleder er ledig og det finnes en styrenestleder, skal styrenestleder umiddelbart rykke opp til vervet styreleder med midlertidig virkning frem til Styret evt. har supplert seg med en ny styreleder.

\section{Styremøter}
Det avholdes styremøte dersom styreleder, styrenestleder eller minst to styremedlemmer kaller inn til det. Styret er vedtaksdyktig under styremøter når \nicefrac{3}{5} av Styrets medlemmer og styreleder eller styrenestleder er tilstede. Ved saksbehandling gjelder absolutt flertall av alle Styrets medlemmer. Ved stemmelikhet har leder dobbeltstemme.

\section{Distribuert saksbehandling}
Styret kan foruten styremøter behandle saker gjennom distribuert saksbehandling. Styret er vedtaksdyktig under distribuert saksbehandling dersom alle styremedlemmer er varslet og minst \nicefrac{3}{4} av Styrets medlemmer har avgitt stemmen sin innen 24 timer. Dersom noen av Styrets medlemmer krever det så skal saken utsettes til neste styremøte.

\section{Signaturrett}
Styreleder, styrenestleder og økonomiansvarlig har signaturrett for foreningen hver for seg. Styret kan også tildele andre personer prokura.
