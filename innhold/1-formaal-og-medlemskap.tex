\chapter{Formål og medlemskap}

\section{Navn og formål}
Casual Gaming (forkortet CaG) er en frivillig student- og ungdomsforening basert i Trondheim med formålet å fremme interesse og engasjement for data og spill. Casual Gaming er partipolitisk og religiøst uavhengig.

\section{Medlemmer av Casual Gaming}
Som medlemmer i Casual Gaming regnes alle som har betalt kontingenten for inneværende år.

\section{Medlemsfordeler}
Medlemmer og styremedlemmer har under ingen omstendigheter mulighet til å låne, eller på annen måte få utbytte av Casual Gamings økonomiske kapital.

\subsection{Utstyrslån}
Medlemmer i Casual Gaming kan etter godkjenning fra leder av Tech låne utstyr som Casual Gaming eier. Dette gjelder svitsjer, kabler, servere og lignende. Under utlån er låner økonomisk ansvarlig for utstyret, og det må leveres i god tid før offisielle arrangementer.

\section{Medlemsrettigheter}
Medlemmer av Casual Gaming kan kreve innsyn i organisasjonens regnskap og kan få innsyn i ikke-konfidensielle vedtak. Medlemmer som har utført et arbeid for Casual Gaming har rett til å få attest av Styret ved forespørsel.

\section{Utmelding, opphør og ekskludering}
Alle medlemmer kan melde seg ut av organisasjonen ved å melde fra skriftlig til Styret. Utmelding gjelder fra utmeldingsdato. Når en person ikke lenger oppfyller kravene for medlemskap i organisasjonen vil medlemskapet opphøre. Ved grov uaktsomhet eller brudd på norsk lov kan Styret, med $\nicefrac{2}{3}$-flertall, fatte vedtak om å ekskludere et medlem fra organisasjonen.
