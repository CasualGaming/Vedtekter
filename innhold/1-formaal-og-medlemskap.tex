\chapter{Formål og medlemskap}

\emptychapterspacing

\section{Navn og formål}
Casual Gaming (forkortet CaG) er en frivillig student- og ungdomsforening basert i Trondheim med formålet å fremme interesse og engasjement for data og spill. Casual Gaming er partipolitisk og religiøst uavhengig.

\section{Medlemskap}
Som medlemmer i Casual Gaming regnes alle som har betalt kontingenten for inneværende år.

\section{Medlemsrettigheter}
Medlemmer av Casual Gaming har følgende rettigheter:
\begin{enumerate}[a.]
    \item Medlemmer kan kreve innsyn i organisasjonens regnskap og kan få innsyn i ikke-konfiden\-sielle vedtak.
    \item Medlemmer som har utført et arbeid for Casual Gaming har rett til å få attest av Styret ved forespørsel.
    \item Medlemmer kan etter godkjenning fra styret låne utstyr som Casual Gaming eier. Styret kan avslå forespørsler om utlån dersom de ikke regner lånet som trygt og hensiktsmessig. Styret kan delegere ansvaret for utlån. Under utlån er låner økonomisk ansvarlig for utstyret, og det må leveres tilbake i god tid før offisielle arrangementer.
    \item Medlemmer og styremedlemmer har under ingen omstendigheter mulighet til å låne eller på annen måte få utbytte av Casual Gamings økonomiske kapital.
\end{enumerate}

\section{Utmelding, opphør og ekskludering}
Alle medlemmer kan melde seg ut av organisasjonen ved å melde fra skriftlig til styret. Utmelding gjelder fra utmeldingsdato. Når en person ikke lenger oppfyller kravene for medlemskap i organisasjonen vil medlemskapet opphøre. Ved grov uaktsomhet eller brudd på norsk lov kan styret, med \nicefrac{2}{3}-flertall, fatte vedtak om å ekskludere et medlem fra organisasjonen.
