\chapter{Organisasjonsstruktur}

\section{Styret}
Styret er det øverste organet i foreningen mellom årsmøtene. Styrets ansvarsområde går på å drive Casual Gaming som organisasjon mellom årsmøtene.

\subsection{Styrets oppbygning}
Styret består av følgende styreverv:
\begin{itemize}
    \item Leder av Stryret
    \item Nestleder av Styret
    \item Økonomiansvarlig
    \item Leder av Coord
    \item Leder av Info
    \item Leder av Tech
    \item Leder av Game
    \item Leder av Care
\end{itemize}

\subsection{Styreverv}
Et styremedlem innehar sitt verv til ny person har blitt valgt til stillingen. Styret har fullmakt til å omfordele styreverv og å supplere seg selv ved ledige styreverv. Dersom et styreverv står ledig, skal Styret internt tildele ansvaret for vervet innad i Styret. Dersom vervet som styreleder er ledig og det finnes en styrenestleder, skal styrenestleder umiddelbart rykke opp til vervet styreleder med midlertidig virkning frem til Styret evt. har supplert seg med en ny styreleder.

\subsection{Styremøter}
Det avholdes styremøte dersom styreleder, styrenestleder eller minst to styremedlemmer kaller inn til det. Styret er vedtaksdyktig under styremøter når $\nicefrac{3}{5}$ av Styrets medlemmer og styreleder eller neststyreleder er tilstede. Ved saksbehandling gjelder absolutt flertall av alle Styrets medlemmer. Ved stemmelikhet har leder dobbeltstemme.

\subsection{Distribuert saksbehandling}
Styret kan foruten styremøter behandle saker gjennom elektronisk behandling. Styret er vedtaksdyktig under elektronisk behandling dersom alle styremedlemmer er varslet og minst ¾ av Styrets medlemmer har avgitt stemmen sin innen 24 timer. Dersom noen av Styrets medlemmer krever det så skal saken utsettes til neste styremøte.

\subsection{Signaturrett}
Styreleder, styrenestleder og økonomiansvarlig har signaturrett for foreningen hver for seg. Styret kan også tildele andre personer prokura.

\section{Crewet}
Crewet bistår driften av Casual Gamings ansvarsområder og er inndelt i seks grupper som ledes av Styret: Coord, Info, Tech, Game, Care og Event.

\subsection{Crewmedlemskap}
Crewmedlemmer velges gjennom opptak som gjennomføres av Styret. Ved opptak skal crewmedlemmet tilordnes nøyaktig én crewgruppe, men tilordningen kan endres av Styret ved ønske om det. Alle crewmedlemmer skal være medlemmer av Casual Gaming. Crewmedlemskap overføres automatisk til nytt år dersom medlemskapet i Casual Gaming fortsettes. Crewmedlemskap avbrytes med umiddelbar virkning dersom crewmedlemmet gir beskjed om det eller hvis Styret vedtar å utestenge crewmedlemmet. Styreverv medfører automatisk crewstatus som ikke kan avbrytes så lenge vedkommende innehar styrevervet. Styrevervene som ledere for Coord, Info, Tech, Game og Care medfører automatisk tilordning til crewgruppene man er leder før, mens resterende styreverv ikke skal tilordnes noen crewgruppe.

\subsection{Grupper}

\subsubsection{Coord}
Coord har ansvar for å skaffe sponsing og å bistå Styret med koordinering av arrangementene til Casual Gaming. Oppgaver inkluderer blant annet booking av lokaler, søking av støtte, koordinering av vakthold og koordinering av transport.

\subsubsection{Info}
Info er Casual Gamings ansikt utad og har ansvar for PR-arbeid og mediaproduksjon i forbindelse med arrangementene til Casual Gaming. Arbeidet inkluderer annonseringer, anskaffelse av promomateriell, markedsføring, svar i kommunikasjonskanaler, streamproduksjon og sceneproduksjon.

\subsubsection{Tech}
Tech er ansvarlig for utvikling og drift av informasjonssystemene til Casual Gaming og er ansvarlig for planlegging og drift av teknisk infrastruktur under arrangementene til Casual Gaming, deriblant strøm, datanettverk og servere. Gruppen har ansvar for å yte teknisk brukerstøtte under arrangementer. Gruppen har i tillegg ansvar for utlån av utstyr til medlemmer av Casual Gaming.

\subsubsection{Game}
Game er ansvarlig for konkurransene og spillservere som avholdes både under og utenfor arrangementene til Casual Gaming.

\subsubsection{Care}
Care er ansvarlig for drift av kiosk og trivsel for crew og deltagere under Casual Gamings arrangementer. Gruppen har også ansvar for gjennomføring av interne arrangementer.

\subsubsection{Event}
Event består av midlertidige crewmedlemmer som rekrutteres i forbindelse med arrangementer. Gruppens ansvar er å støtte de andre gruppene under rigging og drifting av arrangementet. Etter arrangementet vil gruppen automatisk avskjediges.

\section{Mislighold av verv}
Om en innehaver av et verv misligholder sine arbeidsoppgaver, kan ethvert medlem av Casual Gaming stille mistillitsforslag ovenfor vedkommende. Mistillitsforslaget skal leveres skriftlig til Styret, som skal behandle saken. Ved mistillitsforslag mot et styremedlem blir den anklagede umiddelbart suspendert inntil Styret har kommet med en avgjørelse. Mistillitsforslaget leses opp i Styret, deretter skal den anklagede få en mulighet til å forsvare seg før Styret diskuterer og avgjør saken uten den anklagede til stede. For å beholde et beslutningsdyktig og fungerende Styre vil det kun være mulig å stille mistillitsforslag mot ett styremedlem av gangen. Styret har to uker på å behandle et mistillitsforslag.
