\chapter{Organisasjonsstruktur}

\section{Styret}
Styret er det øverste organet i foreningen mellom årsmøtene. Et styremedlem innehar sitt verv til ny person har blitt valgt til stillingen. Styret har fullmakt til å supplere seg selv ved frafall. Styrets ansvarsområde går på å drive Casual Gaming som organisasjon mellom årsmøtene.

\subsection{Styrets oppbygning}
Styret består av følgende styreverv:
\begin{itemize}
    \item Leder
    \item Nestleder
    \item Økonomiansvarlig
    \item Leder av Coord
    \item Leder av Info
    \item Leder av Tech
    \item Leder av Game
    \item Leder av Game
    \item Leder av Care
\end{itemize}

\subsection{Økonomiansvarlig}
Økonomiansvarlig er et valgfritt styreverv som Styret har fullmakt til å supplere seg med. Hvis det ikke velges en økonomiansvarlig, vil dette ansvaret overdras Styrets leder og nestleder.

\subsection{Vedtaksdyktighet}
Styret er kun vedtaksdyktig når minst $\nicefrac{3}{5}$ av Styrets medlemmer er tilstede.

\subsection{Innkalling til styremøte}
Det avholdes styremøte dersom leder, nestleder eller minst to styremedlemmer kaller inn til det.

\subsection{Signaturrett}
Styreleder, styrenestleder og økonomiansvarlig har signaturrett for foreningen hver for seg. Styret kan også tildele andre personer prokura.

\section{Grupper}
Casual Gaming er i foruten Styret delt opp i crewgrupper som har ansvaret for Casual Gamings ansvarsområder. Disse seks gruppene er Coord, Info, Tech, Game, Care og Event. Hver gruppe, utenom Event, har en leder som har ansvar for gruppens ledelse og økonomi. Event ledes av Styrets nestleder. De øvrige medlemmene velges ut gjennom opptak til hver enkelt gruppe som gjennomføres av Styret.

\subsection{Coord}
Coord har ansvar for koordinering av Casual Gamings aktiviteter. Coord er blant annet ansvarlig for booking av lokale, sponsorer, vakthold og organisering av rydding før, under og etter arrangementet. Coord har i tillegg ansvar for å skaffe til veie premier i alle konkurranser, om det så gjelder sponsorer eller innkjøp.

\subsection{Info}
Info er Casual Gamings ansikt utad og har ansvar for alt PR-arbeid i forbindelse med arrangementene til Casual Gaming. Dette crewet har ansvaret for alt PR­‐arbeid i sammenheng med arrangementene til Casual Gaming. Arbeidet inkluderer, men er ikke begrenset til, bestilling og anskaffelse av plakater, markedsføring, informasjon på nettsider, kommunikasjonskanaler mellom deltakere og alle former for digitale medier.

\subsection{Tech}
Tech er ansvarlige for utvikling og drift av informasjonssystemene til Casual Gaming og er ansvarlig for teknisk infrastruktur, deriblant strøm, datanettverk og servere, under arrangementene til Casual Gaming. Crewet har ansvar for å yte brukerstøtte under arrangement dersom de har kapasitet til dette. Crewet har også ansvaret for utlån av utstyr til medlemmer av Casual Gaming.

\subsection{Game}
Game er ansvarlig for alle konkurransene som blir holdt på arrangementene til Casual Gaming. Game har ansvar for å sette opp konkurranser og påmelding, og legge alt til rette for at konkurransene blir gjennomført så effektivt og enkelt som mulig for deltakerne.

\subsection{Care}
Care er ansvarlig for drift av kiosk og trivsel for crew og deltagere under Casual Gamings arrangementer. Crewet har også ansvar for gjennomføring av interne arrangementer.

\subsection{Event}
Event består av midlertidige crewmedlemmer som rekrutteres i forbindelse med arrangementer. Gruppens ansvar er å støtte de andre gruppene under rigging og drifting av arrangementet. Etter arrangementet vil crewet avskjediges.

\section{Mislighold av verv}
Om en innehaver av et verv misligholder sine arbeidsoppgaver, kan ethvert medlem av Casual Gaming stille mistillitsforslag ovenfor vedkommende. Mistillitsforslaget skal leveres skriftlig til Styret, som skal behandle saken. Ved mistillitsforslag mot et styremedlem blir den anklagede umiddelbart suspendert inntil Styret har kommet med en avgjørelse. Mistillitsforslaget leses opp i Styret, deretter skal den anklagede få en mulighet til å forsvare seg før Styret diskuterer og avgjør saken uten den anklagede til stede. For å beholde et beslutningsdyktig og fungerende Styre vil det kun være mulig å stille mistillitsforslag mot ett styremedlem av gangen. Styret har to uker på å behandle et mistillitsforslag.
