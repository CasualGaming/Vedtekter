\chapter{Organisasjonsstruktur}

\section{Styret}
Styret er det øverste organet i foreningen mellom årsmøtene. Styrets ansvarsområde går på å drive Casual Gaming som organisasjon mellom årsmøtene.

\subsection{Styrets oppbygning}
Styret består av følgende styreverv:
\begin{itemize}
    \item Leder av Stryret
    \item Nestleder av Styret
    \item Økonomiansvarlig
    \item Leder av Coord
    \item Leder av Info
    \item Leder av Tech
    \item Leder av Game
    \item Leder av Care
\end{itemize}

\subsection{Styreverv}
Et styremedlem innehar sitt verv til ny person har blitt valgt til stillingen. Styret har fullmakt til å omfordele styreverv og å supplere seg selv ved ledige styreverv. Dersom et styreverv står ledig, skal Styret internt tildele ansvaret for vervet innad i Styret. Dersom vervet som styreleder er ledig og det finnes en styrenestleder, skal styrenestleder umiddelbart rykke opp til vervet styreleder med midlertidig virkning frem til Styret evt. har supplert seg med en ny styreleder.

\subsection{Styremøter}
Det avholdes styremøte dersom styreleder, styrenestleder eller minst to styremedlemmer kaller inn til det. Styret er vedtaksdyktig under styremøter når $\nicefrac{3}{5}$ av Styrets medlemmer og styreleder eller neststyreleder er tilstede. Ved saksbehandling gjelder absolutt flertall av alle Styrets medlemmer. Ved stemmelikhet har leder dobbeltstemme.

\subsection{Distribuert saksbehandling}
Styret kan foruten styremøter behandle saker gjennom elektronisk behandling. Styret er vedtaksdyktig under elektronisk behandling dersom alle styremedlemmer er varslet og minst ¾ av Styrets medlemmer har avgitt stemmen sin innen 24 timer. Dersom noen av Styrets medlemmer krever det så skal saken utsettes til neste styremøte.

\subsection{Signaturrett}
Styreleder, styrenestleder og økonomiansvarlig har signaturrett for foreningen hver for seg. Styret kan også tildele andre personer prokura.

\section{Crewet}
Casual Gaming er i foruten Styret delt opp i crewgrupper som har ansvaret for Casual Gamings ansvarsområder. Disse seks gruppene er Coord, Info, Tech, Game, Care og Event. Hver gruppe, utenom Event, har en leder som har ansvar for gruppens ledelse og økonomi. Event ledes av Styrets nestleder. De øvrige medlemmene velges ut gjennom opptak til hver enkelt gruppe som gjennomføres av Styret.

\subsection{Grupper}

\subsubsection{Coord}
Coord har ansvar for koordinering av Casual Gamings aktiviteter. Coord er blant annet ansvarlig for booking av lokaler, sponsorer, vakthold og organisering av rydding før, under og etter arrangementet. Coord har i tillegg ansvar for å skaffe til veie premier i alle konkurranser, om det så gjelder sponsorer eller innkjøp.

\subsubsection{Info}
Info er Casual Gamings ansikt utad og har ansvar for alt PR-arbeid i forbindelse med arrangementene til Casual Gaming. Arbeidet inkluderer, men er ikke begrenset til, bestilling og anskaffelse av plakater, markedsføring, informasjon på nettsider, kommunikasjonskanaler mellom deltakere og alle former for digitale medier.

\subsubsection{Tech}
Tech er ansvarlig for utvikling og drift av informasjonssystemene til Casual Gaming og er ansvarlig for teknisk infrastruktur, deriblant strøm, datanettverk og servere, under arrangementene til Casual Gaming. Crewet har også ansvar for å yte brukerstøtte under arrangement dersom de har kapasitet til dette. Crewet har i tillegg ansvar for utlån av utstyr til medlemmer av Casual Gaming.

\subsubsection{Game}
Game er ansvarlig for alle konkurransene som blir holdt under arrangementene til Casual Gaming. Game har ansvar for å sette opp konkurranser og påmelding, og legge alt til rette for at konkurransene blir gjennomført så effektivt og enkelt som mulig for deltakerne.

\subsubsection{Care}
Care er ansvarlig for drift av kiosk og trivsel for crew og deltagere under Casual Gamings arrangementer. Crewet har også ansvar for gjennomføring av interne arrangementer.

\subsubsection{Event}
Event består av midlertidige crewmedlemmer som rekrutteres i forbindelse med arrangementer. Gruppens ansvar er å støtte de andre gruppene under rigging og drifting av arrangementet. Etter arrangementet vil crewet avskjediges.

\section{Mislighold av verv}
Om en innehaver av et verv misligholder sine arbeidsoppgaver, kan ethvert medlem av Casual Gaming stille mistillitsforslag ovenfor vedkommende. Mistillitsforslaget skal leveres skriftlig til Styret, som skal behandle saken. Ved mistillitsforslag mot et styremedlem blir den anklagede umiddelbart suspendert inntil Styret har kommet med en avgjørelse. Mistillitsforslaget leses opp i Styret, deretter skal den anklagede få en mulighet til å forsvare seg før Styret diskuterer og avgjør saken uten den anklagede til stede. For å beholde et beslutningsdyktig og fungerende Styre vil det kun være mulig å stille mistillitsforslag mot ett styremedlem av gangen. Styret har to uker på å behandle et mistillitsforslag.
