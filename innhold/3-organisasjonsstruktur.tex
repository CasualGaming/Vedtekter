\chapter{Organisasjonsstruktur}

\section{Styret}
Styret er det øverste organet i foreningen mellom årsmøtene. Styret består av styremedlemmer som blir valgt av årsmøtet samt økonomiansvarlig som velges av styret på første styremøte for året. Et styremedlem innehar sitt verv til ny person har blitt valgt til stillingen. Styret har fullmakt til å supplere seg selv ved frafall. Styrets ansvarsområde går på å drive Casual Gaming som organisasjon mellom årsmøtene. Dette innebærer alt som ikke omhandler arrangementene som undergruppene har ansvar for.

\subsection{Styrets oppbygning}
Styret skal bestå av følgende medlemmer:
\begin{itemize}
    \item Leder
    \item Nestleder
    \item Økonomiansvarlig
    \item Leder av Coord
    \item Leder av Info
    \item Leder av Tech
    \item Leder av Game
    \item Leder av Game
    \item Leder av Care
\end{itemize}

Medlemmene av styret–med unntak av økonomiansvarlig–velges inn av årsmøtet. Alle styrets valgte medlemmer–med unntak av leder og nestleder–velges samtidig til leder av undergruppen vedkommende stiller til valg for.

\subsection{Vedtaksdyktighet}
Styret er kun vedtaksdyktig når minst 6 av styrets medlemmer er tilstede.

\subsection{Innkalling til styremøte}
Det avholdes styremøte dersom leder, nestleder eller minst to styremedlemmer kaller inn til det.

\subsection{Tegningsrett}
Økonomiansvarlig, leder og nestleder har signaturrett for foreningen. Styret kan også tildele andre personer disposisjons- og tegningsrett.

\section{Grupper}
Casual Gaming er foruten Styret delt opp i fem grupper som har ansvaret for Casual Gamings aktiviteter. Disse fem gruppene er Coord, Info, Tech, Game og Care. Hver gruppe har en leder som har ansvar for gruppens ledelse og økonomi. De øvrige medlemmene velges ut gjennom opptak til hver enkelt gruppe. Dette opptaket blir organisert og gjennomført av Styret.

\subsection{Coord}
Coord er en koordineringsgruppe som har ansvaret for å koordinere de andre crewene for å gjennomføre arrangement så effektivt som mulig. Coord er blant annet ansvarlig for booking av lokale, sponsorer, vakthold og organisering av rydding før, under og etter arrangementet. Coord har i tillegg ansvar for å skaffe til veie premier i alle konkurranser, om det så gjelder sponsorer eller innkjøp.

\subsection{Info}
Info er Casual Gamings ansikt utad. Dette crewet har ansvaret for alt PR­‐arbeid i sammenheng med arrangementene til Casual Gaming. Arbeidet inkluderer, men er ikke begrenset til, bestilling og anskaffelse av plakater, markedsføring, informasjon på nettsider, kommunikasjonskanaler mellom deltakere og alle former for digitale medier.

\subsection{Tech}
Tech er tekniske ansvarlig for informasjonssystemene, deriblant nettstedene, og arrangementene til Casual Gaming. Under arrangementene bærer crewet ansvaret for at strøm, nett og teknisk infrastruktur blir satt opp og fungerer som det skal. Crewet har ansvar for å yte brukerstøtte under arrangement dersom de har kapasitet til dette. Crewet har også ansvaret for utlån av utstyr til medlemmer av Casual Gaming.

\subsection{Game}
Game er ansvarlig for alle konkurransene som blir holdt på arrangementene til Casual Gaming. Game har ansvar for å sette opp konkurranser og påmelding, og legge alt til rette for at konkurransene blir gjennomført så effektivt og enkelt som mulig for deltakerne.

\subsection{Care}
Care er ansvarlige for drift av kiosk og å skape trivsel for deltagere og crew under LAN. Crewet har også ansvar for gjennomføring av interne arrangementer.

\subsection{Event}
Event er en gruppe med midlertidig crew som rekrutteres i forbindelse med større arrangementer. Gruppens ansvar er å støtte de andre gruppene under rigging og drifting av arrangementet. Etter arrangementet vil crewet avskjediges. Gruppen har ikke egen leder, men ledes av nestleder i Casual Gaming.

\section{Mislighold av verv}
Om en innehaver av et verv misligholder sine arbeidsoppgaver, kan ethvert medlem av Casual Gaming stille mistillitsforslag ovenfor vedkommende. Mistillitsforslaget skal leveres skriftlig til Styret, som skal behandle saken. Ved mistillitsforslag mot et styremedlem blir den anklagede umiddelbart suspendert inntil Styret har kommet med en avgjørelse. Mistillitsforslaget leses opp i Styret, deretter skal den anklagede få en mulighet til å forsvare seg før Styret diskuterer og avgjør saken uten den anklagede til stede. For å beholde et beslutningsdyktig og fungerende Styre vil det kun være mulig å stille mistillitsforslag mot ett styremedlem av gangen. Styret har to uker på å behandle et mistillitsforslag.
